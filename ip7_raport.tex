\documentclass[11pt, a4paper]{article}

\usepackage{polski}
\usepackage{geometry}
	\geometry{margin = 2.5cm}
\usepackage{indentfirst}
\usepackage{xcolor}
\usepackage{bold-extra}
\usepackage[T1]{fontenc}
\usepackage[utf8]{inputenc}
\usepackage{palatino}
\usepackage{graphicx}
\usepackage[unicode, colorlinks]{hyperref}
\usepackage{fancyhdr}
\usepackage{multirow}
\usepackage{listings}
\usepackage[shortlabels]{enumitem}
	\setlist[itemize]{--, itemsep = 0cm}
\usepackage{amsmath}
\usepackage{bm}
\usepackage{subfig}
\usepackage[labelfont = bf]{caption}

\linespread{1.3}

\definecolor{codegray}{rgb}{0.5, 0.5, 0.5}
\definecolor{codepurple}{rgb}{0.58, 0, 0.82}
\definecolor{backcolour}{rgb}{0.95, 0.95, 0.92}

\lstdefinestyle{mystyle},
	frame = single
}

\lstset{
	style = mystyle,
%	language = MATLAB,
	morekeywords = {byte},
	literate = %
		{ą}{{\k{a}}}1
		{Ą}{{\k{A}}}1
		{ć}{{\'c}}1
		{Ć}{{\'{C}}}1
		{ę}{{\k{e}}}1
		{Ę}{{\k{E}}}1
		{ł}{{\l{}}}1
		{Ł}{{\L{}}}1
		{ń}{{\'n}}1
		{Ń}{{\'N}}1
		{ó}{{\'o}}1
		{Ó}{{\'O}}1
		{ś}{{\'s}}1
		{Ś}{{\'S}}1
		{ż}{{\.z}}1
		{Ż}{{\.Z}}1
		{ź}{{\'z}}1
		{Ź}{{\'Z}}1
}

\pagestyle{fancy}
\fancyhf{}
\lhead{IP -- lab. nr 7}
\rhead{Bończyk, Kaźmieruk, Skibiński}
\cfoot{\thepage}

\graphicspath{
	{./img/}
	{./img/1_ar} {./img/1_ma} {./img/2_ar} {./img/2_ma}
	{./img/3} {./img/4} {./img/5} {./img/6}
}




%%%
%%%		DOKUMENT
%%%

\begin{document}



%%%
%%%		TITLEPAGE
%%%

\begin{titlepage}
{\LARGE
\begin{center}
	\begin{figure}[h!]
		\centering
		\includegraphics[width = 0.2\linewidth]{C:/good_folder/nauka/inne/polsl_logo_v2}
	\end{figure}
	
	\vspace{0.25cm}
	
	\textbf{\textsc{Politechnika Śląska}}
	
	\textbf{\textsc{Wydział Automatyki, Elektroniki i Informatyki}}
	
	\vspace{1.5cm}
	
	Laboratorium Identyfikacji Procesów
	
	\vspace{1.5cm}
	
	Ćw. nr 7:
	
	Modele sygnałów stochastycznych.
	
	\vspace{0.5cm}
	
	{\Large Sprawozdanie}
\end{center}
}

\vfill

{\Large
\noindent
\textbf{Data laboratorium:} 21.01.2022 r.

\vspace{0.5cm}

\noindent
\textbf{AiR S2-I/Rob}\\
% \textbf{Sekcja nr 1}:
\textbf{Sekcja 2, podsekcja 1:}

\noindent
\hspace*{0.5cm} Klaudia Bończyk\\
\hspace*{0.5cm} Paweł Kaźmieruk\\
\hspace*{0.5cm} Maksymilian Skibiński\\

% rok 2020/2021, sem. letni, gr. Rob, sekcja nr 1

\vspace{0.5cm}
}

\begin{center}
\phantom{\today{} r.}
\end{center}
\end{titlepage}

\setcounter{page}{2}

\newpage

\section*{Wstęp}
Celem ćwiczeń była identyfikacja modeli ciągów czasowych. Szczególna uwaga została zwrócona na wpływ wartości i liczby parametrów modeli AR, MA i ARMA na przebiegi czasowe, funkcję autokowariancji oraz korelację cząstkową. Zbadany został także wpływ różnicowania na ciąg czasowy.

Na końcu sprawozdania zamieszczony został kod, a przynajmniej jego fragmenty w jakiś sposób udowadniające,
że wykonaliśmy ćwiczenie.

\section*{Zadanie 1}
Wygenerowana została odpowiedź dla ciągu AR(1):
\[
	y(i) = \frac{1}{1 + a_1 \ z^{-1}} \ e(i)
\]
dla różnych wartości parametru $a_1$. Otrzymane przebiegi czasowe zostały zamieszczone na 
rys.~\ref{fig:zd1_ar}.

\begin{figure}[p!]
	\centering
	
	\subfloat[$a_1 = -0.8$]{%
		\includegraphics[width = 0.45 \linewidth]{zd1_ar_a_-0.8}%
	}%
	\hfill%
	\subfloat[$a_1 = -0.1$]{%
		\includegraphics[width = 0.45 \linewidth]{zd1_ar_a_-0.1}%
	}%
	
	\subfloat[$a_1 = 0.1$]{%
		\includegraphics[width = 0.45 \linewidth]{zd1_ar_a_0.1}%
	}%
	\hfill%
	\subfloat[$a_1 = 0.8$]{%
		\includegraphics[width = 0.45 \linewidth]{zd1_ar_a_0.8}%
	}%
	
	\subfloat[$a_1 = 0.995$]{%
		\includegraphics[width = 0.45 \linewidth]{zd1_ar_a_0.995}%
	}%
	\hfill%
	\subfloat[$a_1 = 2$]{%
		\includegraphics[width = 0.45 \linewidth]{zd1_ar_a_2}%
	}%
	
	\caption{Przebiegi czasowe w zależności od wartości $a_1$}
	\label{fig:zd1_ar}
\end{figure}

\subsection*{Obserwacje}
\begin{itemize}
\item Wraz ze wzrostem wartości bezwzględnej parametru $a_1$ wzrasta ogółem amplituda sygnału wyjściowego -- zarejstrowane wartości ekstremalne będą coraz większe.
\item Wraz ze wzrostem wartości bezwzględnej parametru $a_1$ wzrasta także wariancja sygnału.
\item Gdy $|a_1| > 1$ model jest niestabilny -- odpowiedź układu nieprzerwanie rośnie.
\end{itemize}

Ciekawiej jest z wpływem znaku parametru $a_1$ na otrzymywaną odpowiedź.
Porównując tak ,,na oko'' przebiegi otrzymywane dla wartości parametru $a_1$ rózniących się tylko znakiem, można zauważyć, że sygnał dla wartości $a_1$ ze znakiem dodatnim podlega dużo większym chwilowym zmianom. By zbadać zjawisko pod lupą możemy przekształcić równanie opisujące wyjście $y(i)$ tak, by zawierało same wartości pobudzenia $e(i)$. Kilka pierwszych takich przekształceń zostało zamieszczonych w tabeli poniżej.
\begin{table}[htbp!]
	\centering
	\begin{tabular}{c||l|l}
		$i$ & $y(i)$, [$a_1 = 0.9$] & $y(i)$, [$a_1 = -0.9$] \\
		\hline
		0 &
			$e_0$ &
			$e_0$ \\
		1 &
			$e_1 - 0.9 e_0$ &
			$e_1 + 0.9 e_0$ \\
		2 &
			$e_2 - 0.9 e_1 + 0.81 e_2$ &
			$e_2 + 0.9 e_1 + 0.81 e_2$ \\
		3 &
			$e_3 - 0.9 e_2 + 0.81 e_1 - 0.729 e_0$ &
			$e_3 + 0.9 e_2 + 0.81 e_1 + 0.729 e_0$ \\
		4 &
			$e_4 - 0.9 e_3 + 0.81 e_2 - 0.729 e_1 + 0.6561 e_0$ &
			$e_4 + 0.9 e_3 + 0.81 e_2 + 0.729 e_1 + 0.6561 e_0$ \\
		$\vdots$ &
			$\vdots$ & $\vdots$
	\end{tabular}
	\caption{Zależność $y(i)$ od $e(i)$ w chwilach przeszłych}
	\label{tbl:y_e}
\end{table}

W zależności od znaku $a_1$ równania różnią się tylko znakami przy kolejnych $e_i$: dla ujemnego $a_1$ wszystkie znaki to plusy, natomiast dla $a_1$ dodatniego znaki zmieniają się na przemian i co więcej dla kolejnych $y_i$ te same wartości $e_i$ z przeszłości są brane ze znakami na przemian.

Jeśli takie modele pobudzimy skokami jednostkowymi zobaczymy, że model dla ujemnego $a_1$ zwróci
odpowiedź aperiodyczną, a model dla dodatniego $a_1$ odpowiedź oscylacyjną.

Na rysunku poniżej widnieje porównanie przebiegów otrzymanych dla $a_1 = \pm 0.9$ oraz wykres przedstawiający róznice pomiędzy kolejnymi wartościami $y_i$ w wartościach bezwględnych, czyli:
$\Delta y(i) = |y(i) - y(i-1)|$.
\begin{figure}[htbp!]
	\centering
	
	\subfloat[przebiegi czasowe]{%
		\includegraphics[width = 0.45 \linewidth]{zd1_a_0.9_y}%
	}%
	\hfill%
	\subfloat[różnice pomiędzy kolejnymi $y_i$]{%
		\includegraphics[width = 0.45 \linewidth]{zd1_a_0.9_dy}%
	}%
	
	\caption{Porównanie przebiegów czasowych dla $a_1 = \pm 0.9$}
	\label{fig:zd1_a_09}
\end{figure}

\section*{Zadanie 2}
Zgodnie z poleceniem przyjmijmy, że $a_1 = 0.8$ i wyznaczmy wariancję otrzymanego sygnału oraz pierwszych kilka wartości funkcji autokorelacji. Przekształcenia prowadzące do otrzymania kolejnych wyrazów funkcji autokowariancji znajdują się w skrypcie. Dla modelu AR(1) otrzymujemy:
\[
	\gamma_i =
	\begin{cases}
		\frac{\lambda^2}{1 - a_1^2} & \text{dla} \ i = 1 \\
		-a_1 \ \gamma_{i-1} & \text{dla} \ i > 1
	\end{cases}
\]

Wzór pozwalający obliczyć wariancję to:
\[
	\text{var}[X] = \text{E}[(X - \mu)^2]
\]

Zgodnie z przekształceniami w skrypcie: $\mu = 0$, a zatem wzór uprości się do postaci takiej samej
co przy obliczaniu wartości funkcji autokowariancji dla $k=0$. Wariancja zatem wynosi:
\[
	\text{var}[y(i)] = \frac{\lambda^2}{1 - a_1^2}
\]

Na rys.~\ref{fig:zd2_ar} zostało zamieszczone porównanie wartości wyznaczonych analitycznie z
wartościami obliczonymi przy pomocy polecenia \texttt{xcorr} na podstawie wygenerowanego
sygnału.
\begin{figure}[p!]
	\centering
	
	\subfloat[$N = 1000$: $k$ od 0 do 10]{%
		\includegraphics[width = 0.45 \linewidth]{zd2_ar_gamma_n_1000_img1}%
	}%
	\hfill%
	\subfloat[$N = 1000$: $k$ od 0 do 100]{%
		\includegraphics[width = 0.45 \linewidth]{zd2_ar_gamma_n_1000_img2}%
	}%
	
	\subfloat[$N = 10000$: $k$ od 0 do 10]{%
		\includegraphics[width = 0.45 \linewidth]{zd2_ar_gamma_n_10000_img1}%
	}%
	\hfill%
	\subfloat[$N = 10000$: $k$ od 0 do 100]{%
		\includegraphics[width = 0.45 \linewidth]{zd2_ar_gamma_n_10000_img2}%
	}%
	
	\subfloat[$N = 100000$: $k$ od 0 do 10]{%
		\includegraphics[width = 0.45 \linewidth]{zd2_ar_gamma_n_100000_img1}%
	}%
	\hfill%
	\subfloat[$N = 100000$: $k$ od 0 do 100]{%
		\includegraphics[width = 0.45 \linewidth]{zd2_ar_gamma_n_100000_img2}%
	}%
	
	\caption{Funkcja autokowariancji w zależności od liczby próbek $N$}
	\label{fig:zd2_ar}
\end{figure}

W tabeli~\ref{tbl:zd2_ar} porównane zostały wartości wariancji obliczonej analitycznie z wyznaczoną
poleceniem \texttt{var} na podstawie wygenerowanych danych.
\begin{table}[htbp!]
	\centering
	\begin{tabular}{c||c|c}
		$N$ & $\text{var}_{xcorr}$ & $\text{var}_{analitycznie}$ \\
		\hline
		1000 & 3.3279 & 2.7778 \\
		10000 & 2.8311 & 2.7778 \\
		100000 & 2.7754 & 2.7778 \\
	\end{tabular}
	\caption{Wariancja w zależności od liczby próbek $N$}
	\label{tbl:zd2_ar}
\end{table}

Wniosek jest właściwie jeden: czym więcej próbek tym lepiej obliczenia wykonane na danych pomiarowych będą pokrywały się z wartościami wyznaczonymi w sposób analityczny.

\section*{Zadanie 3}
Poprzednie dwa zadania zostały wykonane dla modelu MA(1):
\[
	y(i) = (1 + c_1 \ z^{-1}) \ e(i)
\]

Na rys.~\ref{fig:zd3_ma} przedstawiony został wpływ wartości $c_1$ na przebiegi czasowe $y(i)$.

\begin{figure}[p!]
	\centering
	
	\subfloat[$c_1 = -0.8$]{%
		\includegraphics[width = 0.45 \linewidth]{zd1_ma_c_-0.8}%
	}%
	\hfill%
	\subfloat[$c_1 = -0.1$]{%
		\includegraphics[width = 0.45 \linewidth]{zd1_ma_c_-0.1}%
	}%
	
	\subfloat[$c_1 = 0.1$]{%
		\includegraphics[width = 0.45 \linewidth]{zd1_ma_c_0.1}%
	}%
	\hfill%
	\subfloat[$c_1 = 0.8$]{%
		\includegraphics[width = 0.45 \linewidth]{zd1_ma_c_0.8}%
	}%
	
	\subfloat[$c_1 = 0.995$]{%
		\includegraphics[width = 0.45 \linewidth]{zd1_ma_c_0.995}%
	}%
	\hfill%
	\subfloat[$c_1 = 2$]{%
		\includegraphics[width = 0.45 \linewidth]{zd1_ma_c_2}%
	}%
	
	\caption{Przebiegi czasowe w zależności od wartości $c_1$}
	\label{fig:zd3_ma}
\end{figure}

\subsection*{Obserwacje}
Wraz ze wzrostem modułu wartości $c_1$ wzrasta wariancja oraz wartości ekstremalne zarejstrowane w danych.
\\

\subsection*{Funkcja autokowariancji}
Wyprowadźmy wartości funkcji autokowariancji dla modelu MA(1). Model ciągu w postaci równania
różnicowego to:
\[
	y(i) = e(i) + c_1 e(i-1)
\]

Wartość średnia $\bar{y}$:
\begin{align*}
	\bar{y} = \text{E}[y(i)] &= \text{E}[e(i)] + c_1 \text{E}[e(i-1)] \\
	\text{E}[y(i)] &= 0 + c_1 \cdot 0 = 0
\end{align*}

Zatem, tak jak w przypadku modelu AR(1), wzór na funkcję autokowariancji się skraca do postaci:
\[
	\Gamma(y,k) = \text{E}[y(i)y(i-k)]
\]

\paragraph{k = 0:}
\begin{align*}
	\gamma_0 &=
		\text{E}[y^2(i)] = \text{E}[e^2(i)] + 2 c_1 \text{E}[e(i)e(i-1)] + c_1^2 \text{E}[e^2(i-1)] \\
	\gamma_0 &=
		\lambda^2 + 0 + c_1^2 \lambda^2 = (1 + c_1^2) \lambda^2
\end{align*}

\paragraph{k = 1:}
\begin{align*}
	\gamma_1 &=
		\text{E}[y(i)y(i-1)] = \text{E}[e(i)y(i-1)] + c_1 \text{E}[e(i-1)y(i-1)] \\
	\gamma_1 &=
		0 + c_1 \left(
			\text{E}[e^2(i-1)] + c_1 \text{E}[e(i-1)e(i-2)] 
			\right) \\
	\gamma_1 &=
		0 + c_1 \left(
			\lambda^2 + 0
			\right) = c_1 \lambda^2
\end{align*}


\paragraph{k = 2:}
\begin{align*}
	\gamma_2 &=
		\text{E}[y(i)y(i-2)] = \text{E}[e(i)y(i-2)] + c_1 \text{E}[e(i-1)y(i-2)] \\
	\gamma_2 &=
		0 + 0 = 0 \\
\end{align*}

\paragraph{k = 3:}
\begin{align*}
	\gamma_3 &=
		\text{E}[y(i)y(i-3)] = \text{E}[e(i)y(i-3)] + c_1 \text{E}[e(i-1)y(i-3)] \\
	\gamma_3 &=
		0 + 0 = 0
\end{align*}

\paragraph{Zatem:}
\[
	\gamma_i = \begin{cases}
		(1 + c_1^2) \lambda^2 &
			\text{dla} \ i = 0 \\
		c_1 \lambda^2 &
			\text{dla} \ i = 1 \\
		0 &
			\text{dla} \ i \ge 2 \\
	\end{cases}
\]

Przyjmijmy, że $c_1 = 0.8$.
Porównanie wartości wyznaczonych analitycznie z wartościami otrzymanymi przy pomocy polecenia \texttt{xcorr} dla wygenerowanych danych zostały zamieszczone na rys.~\ref{fig:zd3_ma_2}. W tabeli~\ref{tbl:zd3_ma} porównane zostały wariancje: wyznaczona analitycznie i wyznaczona na podstawie danych przy użyciu \texttt{var}.
\begin{table}[htbp!]
	\centering
	\begin{tabular}{c||c|c}
		$N$ & $\text{var}_{xcorr}$ & $\text{var}_{analitycznie}$ \\
		\hline
		1000 & 1.6838 & 1.6400 \\
		10000 & 1.6541 & 1.6400 \\
		100000 & 1.6434 & 1.6400 \\
	\end{tabular}
	\caption{Wariancja w zależności od liczby próbek $N$}
	\label{tbl:zd3_ma}
\end{table}

\begin{figure}[p!]
	\centering
	
	\subfloat[$N = 1000$: $k$ od 0 do 10]{%
		\includegraphics[width = 0.45 \linewidth]{zd2_ma_gamma_n_1000_img1}%
	}%
	\hfill%
	\subfloat[$N = 1000$: $k$ od 0 do 100]{%
		\includegraphics[width = 0.45 \linewidth]{zd2_ma_gamma_n_1000_img2}%
	}%
	
	\subfloat[$N = 10000$: $k$ od 0 do 10]{%
		\includegraphics[width = 0.45 \linewidth]{zd2_ma_gamma_n_10000_img1}%
	}%
	\hfill%
	\subfloat[$N = 10000$: $k$ od 0 do 100]{%
		\includegraphics[width = 0.45 \linewidth]{zd2_ma_gamma_n_10000_img2}%
	}%
	
	\subfloat[$N = 100000$: $k$ od 0 do 10]{%
		\includegraphics[width = 0.45 \linewidth]{zd2_ma_gamma_n_100000_img1}%
	}%
	\hfill%
	\subfloat[$N = 100000$: $k$ od 0 do 100]{%
		\includegraphics[width = 0.45 \linewidth]{zd2_ma_gamma_n_100000_img2}%
	}%
	
	\caption{Funkcja autokowariancji w zależności od liczby próbek $N$}
	\label{fig:zd3_ma_2}
\end{figure}

\subsection*{Obserwacje}
Wniosek jest taki sam jak poprzednio: czym więcej próbek $N$ danych pomiarowych tym mniejsza różnica
pomiędzy wartościami wyznaczonymi na podstawie danych z tymi uzyskanymi analitycznie.

\subsection*{Różnica pomiędzy modelami AR, a MA}
Jak łatwo można zauważyć, modele AR i MA różnią się funkcjami autokorelacji. Dla modelu MA, dla przesunięcia: $k \ge nC$, gdzie przez $nC$ rozumiemy stopień modelu MA($nC$), autokorelacja przyjmuje wartości zerowe -- całkowicie zerowe.
W przypadku modelu AR wartości dążą do zera, ale nie ma wyraźnego skoku do wartości zerowej.

\section*{Zadanie 4}
Dany jest ciąg $y(i)$:
\[
	y(i) = \frac{1}{1 - 2.5 z^{-1} + 2 z^{-2} -0.5 z^{-3}} \ e(i)
\]

Po pobudzeniu modelu otrzymujemy przebieg widoczny na rys.~\ref{fig:zd4_1}.
\begin{figure}[htbp!]
	\centering
	\includegraphics[width = 0.45 \linewidth]{zd4_y}
	\caption{Przebieg czasowy}
	\label{fig:zd4_1}
\end{figure}

Przebieg ten różni się znacznie od poprzednio obserwowanych. Jego wartość średnia ulega zmianom.
Poza tym, poprzednio łatwo zauważalne było, że model pobudzany był losowymi wartościami (białym szumem), widać było jakim mocnym zmianom ulegało wyjście $y(i)$ z próbki na próbkę, a teraz tej losowości tak łatwo nie widać.

Wielomian $A(z^{-1})$ zapisany w postaci iloczynowej to:
\[
	A(z^{-1}) = -0.5 (x - 2) (x - 1)^2
\]

Pierwiastki równania $A(z^{-1}) = 0$ to:
\[
	p_1 = 2, \quad p_2 = p_3 = 1,
\]
więc model jest niestabilny, gdyż pierwiastki $p_2$, $p_3$ leżą na okręgu jednostkowym.%
\footnote{Dla modelu stabilnego pierwiastki równania $A(z^{-1}) = 0$ leżą na zewnątrz okręgu jednostkowego, 
ale pierwiastki równania $A(z) = 0$ leżą wewnątrz tego okręgu. Jest to tylko kwestia pewnych przekształceń.}
Niestabilność można łatwo zauważyć, jeśli model pobudzimy skokiem jednostkowym, co zamieszczamy na
rys.~\ref{fig:zd4_unstable}.
\begin{figure}[htbp!]
	\centering
	\includegraphics[width = 0.45 \linewidth]{zd4_step}
	\caption{Pobudzenie skokiem jednostkowym}
	\label{fig:zd4_unstable}
\end{figure}

Na rys.~\ref{fig:zd4_d} widać przebiegi czasowe (otrzymane w wyniku pobudzenia białym szumem) po różnicowaniu ciągu.
\begin{figure}[htbp!]
	\centering
	
	\subfloat[różnicowanie pierwsze]{%
		\includegraphics[width = 0.45 \linewidth]{zd4_y_1d}%
	}%
	\hfill%
	\subfloat[różnicowanie drugie]{%
		\includegraphics[width = 0.45 \linewidth]{zd4_y_2d}%
	}%	

	\caption{Różnicowanie ciągu}
	\label{fig:zd4_d}
\end{figure}

\begin{figure}[htbp!]
	\centering
	\includegraphics[width = 0.45 \linewidth]{zd4_y_v2}
	\caption{Użycie modelu analogicznego}
	\label{fig:zd4_inny}
\end{figure}

Różnicowanie ciągu pozwoliło usunąć zmiany wartości średniej. Po drugim zróżnicowaniu przebieg czasowy
już przypomina te, które widzieliśmy w poprzednich zadaniach.

To co uzyskaliśmy w wyniku różnicowania ciągu można także uzyskać korzystając z modelu ARMA lub uproszczonego modelu AR, gdyż różnicowanie usunęło te bieguny, które czynił układ niestabilnym.

Jeśli użyjemy modelu:
\[
	y(i) = \frac{(1 - z^{-1})^2}{-0.5 (z^{-1} - 2) (z^{-1} - 1)^2} \ e(i) = \frac{1}{1 -0.5 z^{-1}} \ e(i)
\]
to w rezultacie po pobudzeniu tego modelu tym samym pobudzeniem otrzymamy te same przebiegi co widać na
rys.~\ref{fig:zd4_inny}.

\section*{Zadanie 5}
Dla ciągów danych wygenerowanych dla kilku różnych modeli na jednym wykresie przedstawione zostały:
unormowana funkcja autokowariancji oraz funkcja korelacji cząstkowej. Funkcja korelacji cząstkowej była
wyliczana poprzez identyfikację modeli AR($k$) o coraz większym stopniu $k$. Korelacja cząstkowa dla $k$ to $a_k$.

Na rysunkach~\ref{fig:zd5_1} i \ref{fig:zd5_2} zamieszczone zostały wartości wspomnianych funkcji

\begin{figure}[p!]
	\centering
	
	\subfloat[AR(1)]{%
		\includegraphics[width = 0.45 \linewidth]{zd5_a_1_c_0}%
	}%
	\hfill%
	\subfloat[AR(2)]{%
		\includegraphics[width = 0.45 \linewidth]{zd5_a_2_c_0}%
	}%
	
	\subfloat[AR(3)]{%
		\includegraphics[width = 0.45 \linewidth]{zd5_a_3_c_0}%
	}%
	\hfill%
	\subfloat[MA(1)]{%
		\includegraphics[width = 0.45 \linewidth]{zd5_a_0_c_1}%
	}%
	
	\subfloat[MA(2)]{%
		\includegraphics[width = 0.45 \linewidth]{zd5_a_0_c_2}%
	}%
	\hfill%
	\subfloat[MA(3)]{%
		\includegraphics[width = 0.45 \linewidth]{zd5_a_0_c_3}%
	}%
	
	\caption{Unormowana funkcja autokowariancji i funkcja korelacji cząstkowej}
	\label{fig:zd5_1}
\end{figure}

\begin{figure}[p!]
	\centering
	
	\subfloat[MA(5)]{%
		\includegraphics[width = 0.45 \linewidth]{zd5_a_0_c_5}%
	}%
	
	\subfloat[ARMA(3, 3)]{%
		\includegraphics[width = 0.45 \linewidth]{zd5_a_3_c_3}%
	}%
	
	\subfloat[ARMA(1, 3)]{%
		\includegraphics[width = 0.45 \linewidth]{zd5_a_1_c_3}%
	}%

	\caption{Unormowana funkcja autokowariancji i funkcja korelacji cząstkowej (ciąg dalszy)}
	\label{fig:zd5_2}
\end{figure}

\subsection*{Obserwacje}
\begin{itemize}
\item Dla ciągów modeli AR($nA$) funkcja korelacji cząstkowej $K_{cz}(k)$ zeruje się dla $k > nA$.
Natomiast funkcja autokowariancji dąży do zera, ale nigdy nie spada do niego gwałtownie.
\item Ciągi MA($nC$) zachowują się odwrotnie: funkcja autokowariancji zeruje się dla $k > nC$, ale
funkcja korelacji cząstkowej dąży do zera wraz ze wzrostem $k$, ale nigdy nie spada do niego skokowo.
\item Modele ARMA mają własności będące połączeniem tych dwóch wymienionych wyżej, ale w najgorszym wariancie: żadna z funkcji nie zeruje się dla konkretnego przesunięcia $k$, natomiast wartości te spokojnie spadają do zera.
\end{itemize}

Trzeba jednak pamiętać, że to jak dobrze będzie widać te własności zależy od ilości danych pomiarowych jak
i od samych wartości parametrów modeli.

\clearpage
\newpage

\section*{Zadanie 6}
Ostatnie zadanie jest sprawdzeniem wcześniej wysnutych wniosków w zadaniu praktycznym -- mamy dane
pomiarowe, a poza tym nic nie wiemy o ich ,,rodzicach''. Na podstawie autokowariancji i korelacji cząstkowej
staramy się poznać ich tożsamość.

\subsection*{Obserwacje}
\begin{enumerate}[a)]
\item Autokowariancja raczej łagodnie zmierza do zera, natomiast dla korelacji cząstkowej widoczny jest gwałtowny spadek. Naszym zdaniem model to AR(5).
\item Korelacja cząstkowa łagodnie dąży do zera, więc na pewno nie jest to model AR. Z autokowariancją
nie jest tak łatwo -- z jednej strony widać gwałtwony spadek gdzieś dla $k = 3$, ale wartości te później
wciąż oscylują ze znaczącym wartościami wokół zera. Danych było raczej niewiele (ok. 2000) w porównaniu do naszych poprzednich badań, więc być może z tego powodu granice nie są tak ostre. Model to MA(3) lub ARMA.
Raczej stawiamy nasze pieniądze na MA(3).
\item Obie funkcje wyraźne łagodnie dążą do zera. Model to ARMA.
\item Korelacja cząstkowa wyraźne i gwałtownie staje się zerowa. Model to AR(4).
\item Znów gwałtowny spadek korelacji cząstkowej, przy łagodnym spadku autokowariancji. Pomimo tego spadku, korelacja cząstkowa nie osiąga tak zerowych wartości jak byśmy tego chcieli. Może to wynikać z niedużej
liczby danych. Mimo wszystko uważamy, że model to AR(2).
\item Sytuacja analogiczna do poprzedniej -- AR(2).
\end{enumerate}

Jak widać zdobyta wiedza pozwoliła nam do pewnego stopnia zaproponować modele zdolne do opisu danych,
ale mimo wszystko nie są to metody perfekcyjne. Po pierwsze w przypadku modeli ARMA trudniej jest określić stopnie (być może to my nie zauważyliśmy po prostu jakichś prawidłowości). Po drugie w przypadku niezbyt dużej liczby danych wszystko się trochę ,,rozmywa'' i trudniej jest ujrzeć gwałtowne spadki i wartości prawie-zerowe istotne przy określaniu stopni.

\begin{figure}[p!]
	\centering
	
	\subfloat[dane z pliku: \texttt{a1.txt}]{%
		\includegraphics[width = 0.45 \linewidth]{zd6_a}%
	}%
	\hfill%
	\subfloat[dane z pliku: \texttt{b1.txt}]{%
		\includegraphics[width = 0.45 \linewidth]{zd6_b}%
	}%
	
	\subfloat[dane z pliku: \texttt{c1.txt}]{%
		\includegraphics[width = 0.45 \linewidth]{zd6_c}%
	}%
	\hfill%
	\subfloat[dane z pliku: \texttt{d1.txt}]{%
		\includegraphics[width = 0.45 \linewidth]{zd6_d}%
	}%
	
	\subfloat[dane z pliku: \texttt{e1.txt}]{%
		\includegraphics[width = 0.45 \linewidth]{zd6_e}%
	}%
	\hfill%
	\subfloat[dane z pliku: \texttt{f1.txt}]{%
		\includegraphics[width = 0.45 \linewidth]{zd6_f}%
	}%
	
	\caption{Unormowana funkcja autokowariancji i funkcja korelacji cząstkowej (dane z plików zamieszczonych na platformie PZE)}
	\label{fig:zd6}
\end{figure}

\clearpage
\newpage

\section*{Kod}
Do realizacji zadań wyrzeźbiliśmy ogromne ilości kodu. Tutaj zamieścimy niektóre z nich, pozwalające uzyskać
część z zamieszczonych wyników. Pozostałe można uzyskać przy drobnych modyfikacjach.

\subsection*{Zadanie 1}
\begin{lstlisting}
clear all; close all; clc;

a_arr = sort([-0.1, -0.8, 0.8, 0.1, 0.995, 2]);
e = randn(1000, 1);

for a = a_arr
    y = filter([1], [1, a], e);
    figure;
    plot(y);
    grid on;
    xlabel('i'); ylabel('y');
    [a, var(y)]
    filename = ['zd1_ar_a_', num2str(a), '.png'];
    exportgraphics(gcf, filename, 'Resolution', 200);
end
\end{lstlisting}

\subsection*{Zadanie 2}
\begin{lstlisting}
clear all; close all; clc;

a = 0.8;
n = 100000;
e = randn(n, 1);
y = filter([1], [1, a], e);

[xc, lags] = xcorr(y, 'biased');

gamma = [];

i_end = 100;

gamma = 1 / (1 - a^2);
for i = 1 : i_end
    gamma(end + 1) = -a * gamma(end);
end

k = 0 : i_end;

figure;
stem(lags, xc, 'o');
hold on;
stem(k, gamma, 'x');
hold off; grid on;
xlabel('k'); ylabel('\gamma_k');
xlim([0, 10]);
legend('xcorr', 'analitycznie');

filename = ['zd1_ar_gamma_n_', num2str(n), '_img1.png'];
% exportgraphics(gcf, filename, 'Resolution', 200);

xlim([0, 100]);
filename = ['zd1_ar_gamma_n_', num2str(n), '_img2.png'];
% exportgraphics(gcf, filename, 'Resolution', 200);
\end{lstlisting}

\subsection*{Zadanie 3}
Zadanie 3 to wykonanie zadań 1 i 2 dla modelu MA(1). By to zrobić potrzebujemy dokonać drobnych modyfikacji.

\subsection*{Zadanie 4}
\begin{lstlisting}
clear all; close all; clc;

% generacja sygnału
A = [1 -2.5 2 -0.5];
e = randn(10000, 1);
y = filter(1, A, e);

figure; plot(y);
grid on; xlabel('i'); ylabel('y(i)');

% pierwsze różnicowanie
y_1d = filter([1 -1], [1], y);

figure; plot(y_1d);
grid on; xlabel('i'); ylabel('y(i)');

% drugie różnicowanie
y_2d = filter([1 -1], [1], y_1d);

figure; plot(y_2d);
grid on; xlabel('i'); ylabel('y(i)');

% ten sam efekt daje użycie modelu ARMA, lub skrócenie biegunów
y_v2 = filter([1 -2 1], A, e);
y_v3 = filter(1, [1 -0.5], e);

figure; plot(y_v2);
grid on; xlabel('i'); ylabel('y(i)');

figure; plot(y_v3);
grid on; xlabel('i'); ylabel('y(i)');

% figure(1); exportgraphics(gcf, 'zd4_y.png', 'Resolution', 200);
% figure(2); exportgraphics(gcf, 'zd4_y_1d.png', 'Resolution', 200);
% figure(3); exportgraphics(gcf, 'zd4_y_2d.png', 'Resolution', 200);
% figure(4); exportgraphics(gcf, 'zd4_y_v2.png', 'Resolution', 200);
% figure(5); exportgraphics(gcf, 'zd4_y_v3.png', 'Resolution', 200);
\end{lstlisting}

\subsection*{Zadanie 5}
\begin{lstlisting}
clear all; close all; clc;

% A = [1 0.8]; C = [1];
% A = [1 0.8 0.6]; C = [1];
% A = [1 0.7 0.3 0.45]; C = [1];
% A = [1]; C = [1 0.5];
% A = [1]; C = [1 -0.5 0.9];
% A = [1]; C = [1 -0.5 0.9 -0.4];
% A = [1]; C = [1 -2 0.7525 0.4998 -0.2004 -0.0505];
% A = [1 -1.3 0.5 -0.056]; C = [1 -0.4 -0.34 0.156];
A = [1 -0.7]; C = [1 -0.4 -0.34 0.156];

e = randn(100000, 1);
y = filter(C, A, e);

[xc, lags] = xcorr(y, 'biased');
xc = xc ./ xc(find(lags == 0));

K_cz = [];
k = 0 : 15;
for i = k
    i;
    sys = armax(y, [i 0]);
    K_cz(end + 1) = sys.A(end);
end

figure;
stem(lags, xc, 'x');
hold on;
stem(k, K_cz, 'o');
hold off;
xlim([0, 15]); xlabel('k'); ylabel('r_k / K_cz(k)');
grid on;
legend('unormowana f. autokowariancji', 'korelacja cząstkowa');

filename = ['zd5_a_', num2str(length(A) - 1), ...
    '_c_', num2str(length(C) - 1), '.png'];
exportgraphics(gcf, filename, 'Resolution', 200);
\end{lstlisting}

\subsection*{Zadanie 6}
Zadanie 6 to wykonanie tego samego co w zadaniu 5, ale dla danych odczytywanych z plików.

\end{document}